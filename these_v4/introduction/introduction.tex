\paragraph{Switched systems}

Switched systems are a sub-class of hybrid systems, extensively used...

CPS... are good: smart houses, safety critical systems, automotive industry, electrical engineering... 

but complex, no general method...

A first sub-class is switched systems described by ODEs. 

Another important model is PDEs, but much more complex.

We are interested in synthesizing controllers for such systems (switched ODEs and PDEs). 
many approaches have been developed,
some are specific for switched systems, others are more general.

\paragraph{Controller synthesis}

Dynamical systems evolve within time. Represented by ODEs.

We want to synthesize controllers in order to ensure some properties such as stability, reachability.



state of the art of switched systems: lyapunov approaches and theoretical approaches,
optimal control through optimization... 
Many results for linear systems. 


Here:  Symbolic control, which uses the computer tool with an automated method for 
(mostly offline) controller synthesis. 


All these approaches present the same drawback, they are all subject to the so called 
{\em curse of dimensionality}, which means that the computational cost of 
the associated algorithms is exponential in the dimension $n$ of the state. 
Approaches looking further in time (sequences of control inputs) add an exponential 
complexity with respect to the number of switching modes and the number 
of time steps considered.


The problem of control synthesis 
for hybrid and switched systems has been widely studied and various tools exist. 
The Multi-Parametric Toolbox (MPT 3.0 \cite{MPT3}) for example solves optimal control problems
using operations on polytopes.
Most approaches make use of Lyapunov or the so-called ``multiple Lyapunov functions''
to solve the problem of
control synthesis for switched systems - see for example \cite{TAB09}. 
The approximate bisimulation approach 
abstracts switched systems under the form
of a discrete model \cite{GIR10a,GIR10b} under certain Lyapunov-based stability conditions.
The latter approach has been implemented in PESSOA \cite{PESSOA} and CoSyMA \cite{CoSyMA}.
The approach used in this paper avoids using Lyapunov functions and relies on the notion
of ``(controlled) invariant'' \cite{BLA99}.


\paragraph{Reachability analysis}

can be done for linear systems with zonotopes

require numerical schemes for nonlinear equations

\paragraph{Model Order Reduction}

catch the behavior of a complex system with few variables

permits the extension to high/infinite dimensional systems

\paragraph{Contributions}

Extension to NL systems with interval arithmetics

Extension to NL systems with very cheap Euler's method

Distributed synthesis (equivalent of domain decomposition)
by taking perturbations into account

High dimensional ODEs with balanced truncation

Guaranteed $L^2$ control of 
PDEs with Galerkin projection in association with ball-based control synthesis.


\section{???}

In this chapter, we introduce the class of systems considered in this thesis, and present 
the possible approaches for synthesizing controllers for such systems, enlightening
the underlying difficulties and limits of current methods. At the end of the chapter, 
we present more precisely an existing method of controller synthesis for a class of
switched systems, first introduced in \cite{???} [MINIMATOR]. Most of 
the following developments rely, or are inspired by this method.  
In the following chapters, we improve this method in different manners, in order to extend its field of 
application and improve its efficiency.


\section{Control of switched systems}


In most controller synthesis approaches, the objective is to 
synthesize a rule $\sigma(\cdot)$ such that, from an initial 
state $x_0 \in \R^n$ at time $t_0$, the system reaches a target state $x_t \in \R^n$, or 
gets as close as possible to this target state.
Because we are in a switched framework, a target state is nearly always not 
exactly attainable. 

Symbolic approaches.


\section{Symbolic control}


Several approaches, available for linear and/or nonlinear systems: 
\begin{itemize}
  \item finite state abstraction by (alternating) approximate bisimulation [PESSOA]
  \item finite state abstraction for monotone systems
 \item $\varepsilon$ bisimulation. [Girard]
  \item Feedback refinement relations. [SCOTS,Reissig,...]
  \item [Sriram]
\end{itemize}

In most cases, need computation of reachable sets

Also, curse of dimensionality, to not scale to dimensions more than $6$.


\section{Reachability analysis}

Notation $Post_u$

Basic zonotopes for linear systems

$z=<c,g>$ where $c$ is the center and $g$ are the generators. Then we have

$Z' = <Ac + B,g>...$

Interval analysis allows handling of nonlinear ODEs

[From Butcher ... to DynIbex].


\section{Problem of $(R,S)$-stability}

$R$ recurrence set

$S$ safety set

Definition of the general problem of $(R,S)$-stability. 

Definition of the general problem of $(R_1,R_2,S)$-reachability. 



\section{reachability and RK methods}

\section{Introduction}
%\subsection{Method of SNR'16 based on Runge-Kutta}
As said in \cite{NL_minimator},
in the methods of symbolic analysis and control of
hybrid systems, the way of representing sets of state values
and computing reachable sets for systems defined by
ordinary differential equations (ODEs) is fundamental
(see, e.g., \cite{Althoff2013a,girard2005reachability}).
An interesting approach appeared recently, based on the
propagation of reachable sets using guaranteed Runge-Kutta
methods with adaptive step size control (see \cite{BMC12,immler2015verified}). In
\cite{NL_minimator} such guaranteed
integration methods are used in the framework of {\em sampled switched systems}.

%The approach n SNR'16 for validated
%simulation is based on the numerical {\em Runge-Kutta} integration method method
%[9], [10].
Given an ODE of the form $\dot{x}(t)=f(t,x(t))$, and a {\em set} of initial values $X_0$,
a symbolic (or ``set-valued'') integration method consists in computing a sequence of
approximations $(t_n, \tilde{x}_n)$ of the solution $x(t; x_0)$ of the ODE
with $x_0\in X_0$ such that $\tilde{x}_n \approx x(t_n; x_{n-1})$.
Symbolic integration methods extend classical {\em numerical} integration methods which correspond to the case where $X_0$ is just a singleton $\{x_0\}$.
The simplest numerical method is Euler's method in which $t_{n+1} = t_n + h$
for some step-size $h$ and $\tilde{x}_{n+1} = \tilde{x}_n + h f(t_n,\tilde{x}_n)$; so
the derivative of $x$ at time $t_n$, $f(t_n, x_n)$, is used as an
approximation of the derivative on the whole time interval. This method is very simple
and fast, but requires small step-sizes $h$.
More advanced
methods coming from the Runge-Kutta family use a few
intermediate computations to improve the approximation
of the derivative. The general form of an explicit $s$-stage
Runge-Kutta formula
of the form $\tilde{x}_{n+1}=\tilde{x}_n+h\Sigma_{i=1}^sb_ik_i$
where $k_i=f(t_n+c_ih, \tilde{x}_n+h\Sigma_{j=1}^{i-1}a_{ij}k_j)$
for $i=2,3,...,s$.
A challenging question
is then to compute a bound on the distance between
the true solution and the numerical solution, i.e.:
$\|x(t_n; x_{n-1}) - x_n\|$. This distance is associated to the {\em local
truncation error} of the numerical method.
In~\cite{NL_minimator}, 
%(following the work of \cite{Nedialkov}),
%[Berz and Makino01,Makino-Berz09,
%Makino and Berz. Taylor models and other validated functional inclusion methods.
%J. Pure and Applied Mathematics, vol. 4, no. 4, 2003.
%Berz and Makino. Verified integration of ODEs and 
%flows using differential algebraic
%methods on high-order Taylor models. Reliable Computing, vol. 4, 1998.],
such a bound is computed using the {\em Lagrange
remainders} of Taylor expansions.
This is achieved using {\em affine arithmetic} \cite{AffineA97}
(by application of the Banach's fixpoint theorem and 
Picard-Lindel\"of operator, see \cite{Nedialkov}). 
In the end, the Runge-Kutta based method of \cite{NL_minimator} is an elaborated method
that requires the use of affine arithmetic, 
%the manipulation of vectors of intervals, 
Picard iteration and 
computation of Lagrange remainder.
%\subsection{One-sided Lipschitz constant}

In contrast, in this paper, we use ordinary arithmetic
(instead of affine arithmetic)
and a basic Euler scheme (instead of Runge-Kutta schemes). We
need neither estimate Lagrange remainders nor perform Picard iteration
in combination with Taylor series.
Our simple Euler-based approach is made possible  by having 
recourse to the notion of {\em one-sided Lipschitz} (OSL) function
\cite{Donchev98}.
This allows us to bound directly the {\em global error},
i.e. the distance between the approximate point~$\tilde{x}(t)$
computed by the Euler scheme
and the exact solution $x(t)$ for all $t\geq 0$
(see Theorem~\ref{th:1}).

{\bf Plan.}
In Section \ref{sec:rw}, we give  details on related work.
In Section \ref{sec:OSL}, we state our main result that bounds
the global error introduced by the Euler scheme in the context of systems
with OSL flows.
In Section~\ref{sec:appl}, we explain how to apply this result
to the synthesis of symbolic control of sampled switched systems.
We give numerical experiments and results
in Section \ref{sec:experiment} for five exampes of the literature,
and compare them with results obtained with the method of \cite{NL_minimator}.
We give final remarks in Section \ref{sec:fr}.
\section{Related work}\label{sec:rw}

Most of the recent work on the symbolic (or set-valued) integration of nonlinear ODEs is based
on the upper bounding of the Lagrange remainders either in the framework of 
Taylor series or Runge-Kutta schemes 
\cite{Althoff2013a,BCD13,BMC12,CAS12,chen2013flow,NL_minimator,Makino2009,report,dit2016validated}.
Sets of states are generally represented as vectors of intervals 
(or ``rectangles'')
and are manipulated  through interval arithmetic \cite{Moore66}
or affine arithmetic \cite{AffineA97}.
%
Taylor expansions with Lagrange remainders are also used in the work
of \cite{Althoff2013a}, which uses ``polynomial zonotopes'' 
for representing sets of states in addition to interval vectors.
%
None of these works uses
the Euler scheme nor the notion of one-sided Lipschitz constant.

In the literature on symbolic integration, the Euler scheme with OSL conditions
is explored in \cite{Donchev98,Lempio95}.
%
Our approach is similar but establishes an {\em analytical} result for the global error of Euler's estimate
(see Theorem \ref{th:1}) rather than analyzing, in terms of complexity,
the speed of convergence to zero, the accuracy and the stability of Euler's method.

In the control literature, OSL conditions
have been recently applied to control and stabilization 
\cite{Abbaszadeh2010,Cai2015},
but do not make use of Euler's method.
%
To our knowledge, our work applies for the first time Euler's scheme
with OSL conditions to the symbolic
control of hybrid systems.


\section{Euler's method/RK methods}


The computation of reachable sets for continuous-time dynamical
systems has been intensively studied during the last decades.
%While
%efficient methods to deal with linear dynamical systems have been
%defined \cite{LeGuernic2009,Bak2017}, the computation of reachable
%sets for non-linear systems is still a challenge.
Most of the methods
to compute the reachable set start from an \textit{initial value
  problem} for a system of \textit{ordinary differential equations} (ODE) defined
by
\begin{equation}
  \label{eq:ivp-ode}
  \dot{x}(t)=f(t,x(t))
  \quad\text{with}\quad x(0)\in X_0 \subset \mathbb{R}^n
  \quad\text{and}\quad t\in[0,t_{\text{end}}]\enspace.
\end{equation}
%Note that a {\em set} of initial values $X_0$ is considered in
%Equation~\eqref{eq:ivp-ode}.
As an analytical solution of
Equation~\eqref{eq:ivp-ode} is usually not computable, numerical
approaches have been considered.
%
A numerical method to solve Equation~\eqref{eq:ivp-ode}, when $X_0$ is
reduced to one value, produces a discretization of time, such that
$t_0 \leqslant \cdots \leqslant t_N = t_{\text{end}}$, and a sequence
of states $x_0$, \dots, $x_{N}$ based on an integration method
%$\Phi(f, x, t,h)$
which starts from an initial value $x_0$ at time
$t_0$ and a finite time horizon~$h$ (the step-size), produces an
approximation $x_{k+1}$ at time $t_{k+1} = t_k + h$, of the exact
solution $x(t_{k+1})$,
%\textit{i.e.}, $x(t_{n+1}; x_n) \approx \Phi(f, x_n,t_n,h)$
for all $k = 0,\dots,N-1$. The simplest numerical
method is Euler's method in which $t_{k+1} = t_k + h$ for some
step-size $h$ and $x_{k+1} = x_k + h f(t_k,x_k)$; so the derivative of
$x$ at time $t_k$, $f(t_k, x_k)$, is used as an approximation of the
derivative on the whole time interval.

%This method is very simple and
%fast, but requires small step-sizes $h$. More advanced methods, as for
%example the Runge-Kutta family, use a few intermediate computations to
%improve the approximation of the derivative.


The global error $error(t)$ at $t=t_0+kh$ is equal to
$\|x(t) -x_{k}\|$.  In case $n=1$, if the solution~$x$  has a bounded second derivative and $f$ is Lipschitz continuous in its second argument, then it
 satisfies:
\begin{equation}\label{eqn:error}
    error(t) \leq {\frac {hM}{2L}}(e^{L(t-t_{0})}-1)
  \end{equation}
%  \eqref{eqn:error}
%    $| GTE | \leq {\frac {hM}{2L}}(e^{L(t-t_{0})}-1)$

where $M$ is an upper bound on the second derivative of $x$ on the given interval and $L$
is the Lipschitz constant of $f$ \cite{atkinson2008introduction}.
\footnote{Such a bound has been used in hybridization methods:
$error(t)=\frac{E_D}{L}(e^{Lt}-1)$ \cite{asarin2007hybridization,chen2016decomposed},
where $E_D$ gives the maximum difference of the derivatives of the original and approximated systems.}\\

%%The classical result on Euler's method is an asymptotic result, saying that the error $\delta_n$ between the estimated result given by Euler's method after step~$n$ and the exact solution, converges to 0 as the number of steps $n=T/h$  tends to infinity (in O($n$)). This holds if the initial error $\delta_0$ is not ``too big''.

In \cite{SNR17}, we gave an upper bound on the global error $error(t)$,
which is more precise than (\ref{eqn:error}).
%under the form
%of a function $e$ (involving polynomials and exponentials) taking as arguments
%time $t$ and the initial error $\delta_{0}$ at time $t=0$.
%
This upper bound makes use of the notion of
{\em One-Sided Lipschitz (OSL)} constant. This notion has been used for the first
time by
\cite{donchev1998stability}
in order to treat ``stiff'' systems of differential equations for which the explicit Euler method is
numerically ``unstable'' (unless the step size is taken to be extremely small).
%have (in the case of linear equations) eigenvalues of very different modulus.
%The stability of systems governed by stiff equations is difficult to establish using only  Lipschitz constants.\footnote{The Euler method can also be numerically unstable, especially for stiff equations, meaning that the numerical solution grows very large for equations where the exact solution does not (Wikipedia).}
%
Unlike Lipschitz constants, OSL constants can be {\em negative},
which express a form of contractivity of the system dynamics.
Even if the OSL constant is positive, it is in practice much lower than
the Lipschitz constant \cite{dahlquist1976error}. The use of OSL thus allows us to obtain a much
more precise upper bound for the global error.
%$|x_n - x(t_n)|$.
%
We also explained in \cite{SNR17} how such a precise estimation of the
global error can be used to synthesize {\em safety controllers} for a special
form hybrid systems, called ``sampled switched systems''.\\

% In this paper,
% we explain how such an Euler-based method can be extended to synthesize safety controllers in a {\em distributed} manner. This allows us to control separately a component using only partial information on the other components. It also allows us to scale up the size of the global systems for which a control can be synthesized.\\



In this paper, we explain how such an Euler-based method can be extended to
synthesize safety controllers in a {\em distributed} manner. This allows us
to control separately a component using only partial information on the other
components. It also allows us to scale up the size of the global systems for
which a control can be synthesized. In order to perform such a distributed
synthesis, we will see the components of the global systems as being {\em interconnected}
(see, \textit{e.g.}, \cite{yang2015lyapunov}),
and use (a variant of) the notions of {\em incremental input-to-state stability ($\delta$-ISS)}
and {\em ISS Lyapunov functions}
\cite{jiang1996lyapunov}
instead of the notion of OSL used in the centralized framework.\\


The plan of the paper is as follows: In Section \ref{sec:background}, we recall the
results of \cite{SNR17} obtained in the centralized framework; in Section \ref{sec:distributed}
we extend these results to the framework of distributed systems;
we then apply the distributed synthesis method to a nontrivial example
(Section~\ref{sec:application}), and conclude in Section~\ref{sec:conc_part4}.




\section{Introduction}
\paragraph{Control of switching systems}
The importance of switching systems has grown up considerably over the
last years because of their ease of implementation for controlling
cyber-physical systems.  A~switching system is a family of
sub-systems, each having its own dynamics, characterized by a
parameter~$u$ whose values range over a finite set~$U$
(see~\cite{liberzon2012switching}). However, when composing
the sub-systems together, the number of modes of the global system
grows exponentially, and the dynamics may become very complex.
%has a number of modes and dynamics which increases exponentially.
It~is therefore essential to design \emph{compositional} analysis techniques
in order to obtain control methods for switching systems
with formal correctness guarantees.
%that give formal guarantees on the correct behaviour of the cyber
%physical systems.

In this paper, we give a symbolic compositional method that allows to
synthesize a control of sampled switching systems that is
guaranteed to satisfy \emph{reachability} and \emph{stability}
properties.
%\fbox{NM: description pas tres precise, je trouve}
The method starts from a (rectangular) target region~$R$ of the state space.
%which is rectangular (i.e.,~a~product of closed intervals of reals).
It~then generates in a backward manner an~increasing sequence of
nested rectangles $\{R^{(i)}\}_{i\geq 0}$ such that any trajectory
issued from~$R^{(i)}$ is guaranteed to reach~$R^{(i-1)}$ in bounded
time.  Stability is achieved by requiring that any trajectory from~$R$
goes back to~$R$ in bounded time.
%Once~${R^{(0)} = R}$ is reached, the~trajectory is guaranteed
%to stay in~$R$ indefinitely (stability).
The method relies on a
simple operation of \emph{tiling} of the rectangles~$R^{(i)}$ in a
finite number of sub-rectangles~(tiles), using a standard operation of
\emph{bisection}.  Although the method works in a backward fashion,
it~does not require to inverse the linear dynamics of the system, 
and does not compute \emph{predecessors} of
symbolic states (tiles): it~computes \emph{successors} using the forward
dynamics.  This~is useful in order to avoid numerical imprecisions,
especially when the dynamics are \emph{contractive}, which happens
often in practical systems (see~\cite{Mitchell07}).

Another contribution of this paper is a technique
of state \emph{over-approximation} which allows 
a distributed control synthesis:
% of local state \emph{over-approximation}
% of a system component, say $2$:
this over-approximation allows sub-system~$\mathcal S_1$ to infer a correct value for
its next local mode~$u_1$ without knowing the exact value of the state
of sub-system~$\mathcal S_2$.
This distributed synthesis method is computationally efficient,
and works in presence of partial observability.
This is at the cost of the performance of the control,
which in this case usually makes the delay for reaching the objective longer
%the trajectories reach the objective area in more steps
than with a centralized approach.

The main part of the paper deals with the \emph{discrete-time}
framework.  The last part explains how the results can be extended to the
\emph{continuous-time} setting.
%\fbox{NM: + continuous, non-linear case}

\paragraph{Related Work}
In symbolic analysis and control-synthesis methods for hybrid systems,
the method of backward reachability and the use of polyhedral symbolic
states, as used here, is classical
(see~e.g.~\cite{asarin2000effective,gillula2011applications}). The~use
of tiling or partitioning the state-space using bisection is also
classical (see~e.g.~\cite{jaulinbook,girard2012low}).  The~main
original contribution of this paper is to give a simple technique of
over-approximation, which allows one component to estimate the
symbolic states of the other components, in~presence of partial
information. This~is similar in spirit to an \emph{assume-guarantee}
(or~\emph{contract-based}) reasoning, where the controller synthesis
for each sub-system assumes that some safety properties are satisfied
by the other
sub-systems~\cite{alur1999reactive,BogomolovFGGPPS14,DallalT15,fribourg2015games,KimAS15,meyer2015safety,Sangiovanni-VincentelliDP12,SmithNO16}.
The~present work is a continuation of~\cite{fribourg2015games}.
In~contrast to~\cite{fribourg2015games}, we~do not need, for the mode
selection of a sub-system, to~blindly explore all the possible modes
selected by the other sub-system.  This~yields a drastic reduction
of the complexity\footnote{This separability technique is made
  possible because the difference equation
  $x_1(t+1)=f_1(x_1(t),x_2(t),u_1)$ (see~Section~\ref{ss:modes}) does
  not involve the control mode~$u_2$.}.  This approach allows us to
treat a real case study, which is intractable using a centralized
approach.  This case study comes from~\cite{larsen2015online}, and we
use the same decomposition of the system into two parts (rooms~$1$
to~$5$ and rooms $6$ to~$11$).  In~contrast to the work
of~\cite{larsen2015online} which uses an on-line and heuristic
approach with no formal guarantees, we~use here an off-line formal
method which guarantees reachability and stability properties.
% citer Asarin Girard Abate Tomlin Tabuada Henzinger
% 
% citer Liberzon (switching systems)
% 
% citer Larsen
% 
% citer Bicchi
%

\paragraph{Implementation}
In the discrete-time setting, with linear (or~affine) mappings,
the~methods of control synthesis both in the centralized and in the
distributed contexts have been integrated to the tool
MINIMATOR~\cite{minimator,fribourg2014finite}, written in
Octave~\cite{octave}.  In~the continuous-time setting (which also allows
nonlinear flows), the~methods have been integrated to the tool DynIBEX
\cite{dynibex,dit2016validated}, written in~C++.  All~the computation
times given in the paper have been performed on a 2.80GHz Intel Core
i7-4810MQ CPU with 8GB of memory.


\subsection{Introduction}
\label{intro}

In this paper, we present a control synthesis method for switched
systems, a class of hybrid systems.  Switched systems have been
recently used in various domains such as automotive industry and, with
noteworthy success, power electronics (\emph{e.g.}, power
converters). Switched systems, continuous-time systems with discrete
switching events, are close to hybrid systems, at the difference that
discrete behaviors are neglected. These systems are merely described
by piecewise dynamics (called modes), switching periodically.
Precisely, at each period, the system is in one and only one mode,
decided by a control rule
\cite{fribourg2014finite,liberzon2012switching}. Moreover, the
considered systems can switch between any two modes
instantaneously. This simplification can be easily by-passed by the
addition of intermediate facticious modes.  We focus here on modes
represented by nonlinear Ordinary Differential Equations (ODEs).
Numerical methods to synthesize a controller for switched systems are
based on simulations of the considered system.

In a previous paper~\cite{NL_minimator}, we proposed an algorithm
based on validated simulation for the synthesis of nonlinear switched
system controllers. Here, we present an improved algorithm which
permits to consider longer patterns (control input sequences) and more
modes, by the help of a better pruning approach.  It also leads to a
strong decrease in computation time. The new algorithm allows us to
handle harder problems than in the previous paper, and to compare our
method to the state-of-the-art.
